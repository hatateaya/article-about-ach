\documentclass[a4paper,11pt]{article}
\usepackage{ctex}

\title{抗胆碱能药物过量所致精神症状及其原理}
\author{作者}
\date{\today}

\begin{document}

\maketitle

乙酰胆碱是人类中枢神经系统的重要神经递质,参与了觉醒、注意、记忆和动机等中枢神经系统的重要功能。抗胆碱能药物是临床常用的一类阻断乙酰胆碱在突触处作用的药物。由于乙酰胆碱参与了中枢神经系统的众多重要功能,过量服用抗胆碱能药物可能导致幻觉、思维障碍、谵妄等严重精神症状。抗胆碱能药物过量常见于误服、自杀、滥用等情况,也偶见于长期使用带抗胆碱活性药物治疗者。该类药物过量往往造成急性的严重精神症状,有的还会伴有后遗症。本文概述了抗胆碱药物过量所致精神症状及其原理。

\section{抗胆碱能药物过量所致精神症状}

急性抗胆碱能药物过量主要是由苯海拉明、莨菪碱类、阿托品等药物或颠茄、曼陀罗等植物过量服用引起的。慢性抗胆碱能药物中毒则可能与某些具有抗胆碱能活性的抗精神病药物、血清素再摄取抑制剂、三环类抗抑郁药以及抗帕金森药等药物有关。

抗胆碱能药物过量会造成身体的和精神的症状。在身体方面,可能会出现肌无力、心动过缓等症状。已有的研究对此状况的身体效应研究的比较清晰,我们主要研究精神方面的症状。

抗胆碱能药物过量(下文称此状况)的急性精神症状主要有以下几类,以患者的感受大小排序:

\begin{itemize}
    \item 低剂量时的焦虑躁狂与高剂量时的情绪淡漠与意志和动力减退。
	\item 磨牙症与不宁腿综合症。
	\item 感知觉障碍
		\begin{itemize}
	    \item 视觉模糊与复视。
	    \item 物体纹理变化与视物扭曲。这种扭曲偶尔是可塑的。
		\item 难以识别物体种类。
		\end{itemize}
	\item 思维障碍与语言表达障碍。
	\item 幻觉
		\begin{itemize}
        \item 此状况所致幻觉包括幻视、幻听及幻触。在睁眼和闭眼的情况下都会出现。
        \item 此状况所致幻觉的内容的形式可能为人、动物等实体,一般比较完整有型,可同其进行交互行为。有些情况下可能会有整个场景的幻觉。幻觉的情节、这些实体的活动、这些事物与实际存在事物的结合等的状况一般较为合理。幻觉的内容大多为日常可能发生的事情。这些内容与人潜意识中考虑、期望或恐惧的一些状况有关,和梦境也有一些相似之处。除有逻辑且完整有型的这些幻觉以外,也可能会有一些琐碎的幻觉,以蜘蛛和“影子人“最为常见。
        \item 此状况所致幻觉的幻视、幻听及幻触常常比较统一,可以产生”看得见,摸得着”的幻觉实体。并且因这些幻觉的完整有型、较有逻辑、符合生活经历等特点,体验者往往对幻觉深信不疑。
		\end{itemize}
	\item 注意障碍、记忆障碍与专注力障碍。
    \item 厄运感。表现为体验到强烈的关于某种厄运即将到来的恐惧。
    \item 梦境增强。表现为梦境的细节更完善,情节更复杂,体验更强烈且真实生动。能否诱发清醒梦暂不清晰。
	\item 谵妄。
\end{itemize}

抗胆碱能药物过量常伴有精神后遗症,主要以注意障碍、记忆障碍、思维障碍、不宁腿综合征为主,可能伴有抑郁焦虑。严重情况下可能有长期的幻觉。

\section{精神症状的原理}

幻觉根据不同类型,分别可能由初级感觉皮层、关联感觉区域等区域异常所致。b.色胺类致幻剂所致幻觉与初级视觉皮层异常有关。c.幻觉可能由感觉输入受损而导致早期感知或痕迹进入意识有关,尤其说视力障碍患者的幻觉。d.幻觉可能与梦境侵入有关,例如入睡前幻觉。多种神经递质或睡眠觉醒节律问题与此有关,尤其是发作性睡病患者的幻觉。



人们已经确定了意识的两个不同组成部分 ,第一个涵盖唤醒-访问-警觉,第二个涵盖心理体验-选择性注意。当意识被描述为明确的、陈述性的或反思性的时,它可以与隐性的、非反思性的、潜意识的或无意识的过程区分开。

REM 睡眠异常也可能暗示REM 侵入清醒状态。REM 睡眠诱导区 40% 的输入来自胆碱能神经元。胆碱类药物引起的脑干脚桥被盖胆碱能细胞的兴奋会诱发 REM 睡眠,而胆碱酯酶抑制剂会缩短 REM 睡眠潜伏期并增加 REM 睡眠持续时间。

\section{中枢胆碱能系统作用}

\subsection{选择性注意}

乙酰胆碱能够增强某些突触的链接强度,从而将相关的信息从潜意识中随机且不相关的信息与一些随机的神经活动噪声中区分开来。在乙酰胆碱水平过度降低的情况下,由于失去了这个区分的功能,因此这些无意义的信息就会进入意识的形成过程而形成幻觉的意识。此外的,由于胆碱类神经元在REM睡眠期活跃,胆碱类物质晶体插入脑内会导致REM睡眠等原因,胆碱能所致幻觉的机理可能也与梦境侵入有关。

胆碱能系统与「选择性注意」系统有关,控制相关活动,是大脑活动和意识的重要组成成分。例如,拮抗毒蕈碱受体药物引发幻觉与低意识,烟碱受体与全麻作用有关。这不仅控制意识的内容,还控制水平或强度。

皮质乙酰胆碱增强神经元信噪比的,皮质中的毒蕈碱受体激活参与限制离散自我报告意识“流”的内容。在没有皮质乙酰胆碱的情况下,当前不相关的内在和感官信息在潜意识层面不断并行处理,从而产生意识。这与通过医疗、娱乐或仪式方式施用抗毒蕈碱药物来诱发幻觉和其他感知障碍的能力一致。

将卡巴胆碱晶体插入内侧前脑、中脑、脑桥或延髓,可诱发猫的 REM 睡眠。

\subsection{REM睡眠}


\section{抗胆碱能药物过量所致精神症状的原理}

\section{作用与展望}

\begin{thebibliography}{99}  

\bibitem{yangchunyong2003}杨春永.苯海拉明中毒二例报告[J].北京医学,2003,(06):406.
\bibitem{shenwenlong1986}沈文龙.苯海拉明所致药源性精神病[J].国外医学.精神病学分册,1986,(01):56.

\end{thebibliography} 

\end{document}
